\documentclass{resume}
\usepackage{hyperref}
\usepackage{lipsum}
\usepackage[left=0.75in,top=0.6in,right=0.75in,bottom=0.6in]{geometry}

\begin{document}
\vspace*{-40pt}

%==== Profile ====%
\vspace*{-10pt}
\begin{center}
	{\Huge \scshape {Apoorva Chintalacheruvu}}\\
	\end{center}
\begin{center}
	Chicago, IL $\cdot$ sgulli3@uic.edu $\cdot$ +1 (312) 721 6147 $\cdot$ \href{https://www.linkedin.com/in/sreevardhan-g-658175119/}{LinkedIn}
\end{center}
\vspace{3mm}

Mechanical engineer with a master’s degree from University of Illinois at Chicago. Very well versed in design and analysis of complex multibody systems. Highly skilled in CAE/CAD tools especially in Solidworks, ANSYS and CATIA.
% Mechanical engineering graduate student at University of Illinois at Chicago interested in design and analysis of complex multibody systems. Skilled in CAE/CAD and looking for opportunities as a design engineer.
%Aspiring design engineer with a graduate degree in Mechanical Engineering at University of Illinois at Chicago. Looking for full time opportunities in design engineering, dynamic simulation, material engineering and HVAC.


%======SKILLS======%
\begin{rSection}{Skills}

\textbf{Engineering:} Mutlibody Dynamics, Finite Element Analysis (FEA), Computer Aided Engineering, GD\&T, CNC Programming, 3D Printing, HVAC
\\\textbf{Software:} Solidworks, ANSYS, CATIA, Inventor, MATLAB, Creo Parametric, AutoCAD, MSC Adams, Minitab, SIGMA/SAMS
\\\textbf{Technologies:} Python, Java, C
% \\\textbf{Courses:} Advanced concepts in Computer Aided Engineering, Applied Stress Analysis, Compressible Flow Theory, Computational Analysis of Multibody Systems, Finite Element Analysis
\end{rSection}


% \header{Experience}

% \textbf{Society of Automotive Engineers Collegiate Chapter — CMRCET}, 2014 – 2017 \hfill \\Founding Member and Treasurer
% \begin{itemize} \itemsep -2pt
% \item Took the initiative of establishing an SAE India collegiate chapter at my college.
% \item Elected as a member of executive committee. Served as treasurer for one year.
% \item In charge of club’s finances. This included raising funds and handling expenses for events and
% competitions.
% \item Organized various technical events to promote club enrollment and participation.\\
% \end{itemize}
%======PROJECTS=====%

\begin{rSection}{Projects}

{\bf Dynamic Analysis of Planar Slider Crank Mechanism}
\vspace{-0.70em}
\begin{itemize}
 \setlength\itemsep{-0.65em}
    \item Used SIGMA/SAMS, a multibody simulation software, to model and simulate the motion and deformation of a rigid slider crank mechanism with a flexible connecting rod.
    \item Created the flexible body using a finite element mesh with a floating frame of reference(FFR).
    \item Transverse deformation of the mid point of the connecting rod was plotted with respect to time.
\end{itemize}
% \\Used SIGMA/SAMS, a multibody simulation software, to model and simulate the motion and deformation of a rigid slider crank mechanism with a flexible connecting rod. Created the flexible body using a finite element mesh with a floating frame of reference(FFR). Transverse deformation of the mid point of the connecting rod was plotted with respect to time.

{\bf CAD Design Project - Wheel hub of an All Terrain Vehicle}
\vspace{-0.70em}
\begin{itemize}
 \setlength\itemsep{-0.65em}
    \item Designed and optimized a wheel hub model for an all-terrain vehicle using CAE and CAD principles.
    \item Used ANSYS workbench and topology optimization tool to determine stress regions and areas of excess material.
    \item Additionally, modeled the suspension system parts such as the brake caliper, brake rotor, knuckle, spindle, upper and lower wishbones and the shock absorber.
\end{itemize}
% \\Designed and optimized a wheel hub model for an all-terrain vehicle using CAE and CAD principles. Used ANSYS workbench and topology optimization tool to determine stress regions and areas of excess material. Additionally, modeled the suspension system parts such as the brake caliper, brake rotor, knuckle, spindle, upper and lower wishbones and the shock absorber.

{\bf A Review of Mechanics and Fatigue/Fracture of Hydrogels}
\vspace{-0.70em}
\begin{itemize}
 \setlength\itemsep{-0.65em}
    \item Performed a review of 20 papers related to the physical and chemical properties of hydrogels. Fracture and fatigue properties of hydrogels were studied in detail.
\end{itemize}
% \\Performed a review of 20 papers related to the physical and chemical properties of hydrogels. Fracture and fatigue properties of hydrogels were studied in detail.

{\bf Baja Student India}
\vspace{-0.70em}
\begin{itemize}
 \setlength\itemsep{-0.65em}
    \item Designed and manufactured a dune buggy. Worked on vehicle chassis design along with transmission, suspensions and electrical systems.
    \item Additionally, handled the finances for this project.
\end{itemize}
% \\Designed and manufactured a dune buggy. Worked on vehicle chassis design along with transmission, suspensions and electrical systems. Additionally, handled the finances for this project.

{\bf Performance analysis of Coated Single Point Cutting Tools in Turning Operation}
\vspace{-0.70em}
\begin{itemize}
 \setlength\itemsep{-0.65em}
    \item Compared the performance of three types of carbide cutting tips for turning operation: CVD coated, nickel coated and uncoated.
    \item Determined the performance of each tip by measuring surface roughness of the job for various cutting parameters like depth of cut, cutting speed and feed rate.
\end{itemize}
% \\Compared the performance of three types of carbide cutting tips for turning operation: CVD coated, nickel coated and uncoated. Determined the performance of each tip by measuring surface roughness of the job for various cutting parameters like depth of cut, cutting speed and feed rate.


% {\textbf{Academic Mini Project - Design and analysis of a Pneumatic Forklift, }CMRCET, 2016} \hfill %\\Project Lead
% \begin{itemize} \itemsep -2pt

% \item Created a rudimentary model of a pneumatic forklift using Creo Parametric 3.0. 
% \item Performed static analysis of said model with standard load conditions using ANSYS 16.\\
% \end{itemize}


\end{rSection}

%==== Education ====%
\begin{rSection}{Education}

\textbf{University of Illinois at Chicago}\hfill Chicago, IL\\
Master of Science --- Mechanical Engineering, GPA: 3.71 \hfill Aug 2018 - May 2020
\vspace{-0.5em}
\begin{itemize}
 \setlength\itemsep{-0.5em}
    \item \textbf{Courses:} Advanced concepts in Computer Aided Engineering, Applied Stress Analysis, Compressible Flow Theory, Computational Analysis of Multibody Systems, Heating Ventilation and Air Conditioning (HVAC). Continuum Mechanics
\end{itemize}
% Master of Science --- Mechanical Engineering, GPA: 3.71 \hfill Aug 2018 - May 2020
% \\\textbf{Courses:} Advanced concepts in Computer Aided Engineering, Applied Stress Analysis, Compressible Flow Theory, Computational Analysis of Multibody Systems, Finite Element Analysis

\textbf{CMR College of Engineering and Technology}\hfill Hyderabad, India\\
\vspace{-0.5em}
Bachelor of Technology --- Mechanical Engineering GPA: 3.8\hfill Sept 2013 - June 2017

\end{rSection}

%=====Leadership======%

%\vspace*{2mm}

%=====GAP YEAR=====%
%\header{Gap Year : June 2017 --- July 2018}
%\begin{itemize}
%    \item Took a year off after undergraduate studies to prepare and plan for my Master's program. 
%\end{itemize}
\end{document}